\documentclass{beamer}

% Common packages
\usepackage[utf8]{inputenc}
\usepackage{graphicx}
\usepackage{booktabs}
\usepackage{hyperref}
\usepackage{listings}
\usepackage{xcolor}

% Choose a theme
\usetheme{Madrid}
\usecolortheme{default}

% Presentation information
\title{Armazenamento de Áudio em Bases de Dados Vetoriais}
\subtitle{UC-Projeto}
\author{Diogo Silva (A100092), Pedro Oliveira (A97686), João Barbosa (A100054)}
\institute{Universidade do Minho}
\date{\today}

\begin{document}

% Title page
\begin{frame}
    \titlepage
\end{frame}

% Table of contents
\begin{frame}{Índice}
    \tableofcontents
\end{frame}

% Sections of your presentation
\section{O que é uma base de dados vetorial?}

\begin{frame}{O que é uma base de dados vetorial?}

    THE CORE OF AI 
    \begin{itemize}
        \item Coleção de dados armazenados como representações matemáticas
        \item Facilitam a memorização de \textit{inputs} anteriores
        \item Permitem o uso de IA para prever \textit{outputs} futuros
        \item Em vez de identificação por \textit{pattern matching} ou \textit{indexing}
        \item Os vetores são comparados com base em similaridade matemática
    \end{itemize}
    
    \vspace{0.5cm}
    \footnotesize{Fonte: Cloudflare}
\end{frame}

\begin{frame}{Exemplo prático para o nosso trabalho}
    \begin{itemize}
        \item Plataforma de streaming de música com recomendações
        \item Algoritmo compara vetores das músicas já ouvidas pelo utilizador
        \item Recomenda músicas com vetores de maior similaridade
        \item Permite que programas façam comparações e identifiquem relações
        \item Facilita a criação de sistemas avançados de IA, como LLMs
    \end{itemize}
\end{frame}

\section{Comparação entre Sistemas de Bases de Dados Vetoriais}

\begin{frame}{Weaviate}
    \begin{itemize}
        \item \textbf{Open-source}
        \item Combina pesquisa vetorial com filtragem estruturada
        \item Suporta múltiplos tipos de dados (texto, imagens, áudio)
        \item API GraphQL flexível
        \item Escalabilidade horizontal
    \end{itemize}
\end{frame}

\begin{frame}{Pinecone}
    \begin{itemize}
        \item Serviço totalmente gerido
        \item Fácil de usar
        \item Alto desempenho
        \item Sistema fechado
        \item Não é open-source
    \end{itemize}
\end{frame}

\begin{frame}{Qdrant}
    \begin{itemize}
        \item Otimizado para pesquisas vetoriais filtráveis
        \item Oferece bom desempenho
        \item Eficiente
        \item Comunidade menor
        \item Menos integrações disponíveis
    \end{itemize}
\end{frame}

\begin{frame}{Chroma}
    \begin{itemize}
        \item Focado em aplicações de LLM e RAG
        \item Simples de integrar e utilizar
        \item Ainda não é tão maduro quanto outras opções
        \item Menos escalável que alternativas
    \end{itemize}
\end{frame}

\section{Porque escolhemos o Weaviate?}

\begin{frame}{Vantagens do Weaviate para o nosso projeto}
    \begin{enumerate}
        \item \textbf{Capacidade de pesquisa híbrida:} Permite pesquisa por similaridade vetorial e filtragem estruturada
        \item \textbf{Flexibilidade de esquema:} Representa ficheiros de áudio com diferentes propriedades e relações
        \item \textbf{API GraphQL:} Simplifica a interação com a base de dados
        \item \textbf{Open-Source:} Comunidade ativa com atualizações regulares
        \item \textbf{Documentação de qualidade:} Agiliza o desenvolvimento
    \end{enumerate}
\end{frame}

\begin{frame}{Comparação com alternativas}
    \begin{itemize}
        \item \textbf{Pinecone:} Mais simples de configurar e escalar automaticamente
        \begin{itemize}
            \item Limitação: não é open-source, menor controlo sobre os dados
        \end{itemize}
        \item \textbf{Milvus:} Alternativa com maior desempenho para bases muito grandes
        \begin{itemize}
            \item Limitação: maior complexidade de configuração
        \end{itemize}
        \item \textbf{Weaviate:} Melhor equilíbrio entre funcionalidades, flexibilidade e facilidade de uso
    \end{itemize}
\end{frame}

\section{Representação de ficheiros de áudio em vetores}

\begin{frame}{Como converter ficheiros de áudio em vetores?}
    \begin{itemize}
        \item Wav2Vec2 ? Framework ig
        \item Conteúdo em desenvolvimento...
        \item Técnicas de processamento de sinal
        \item Extração de características de áudio
        \item Utilização de redes neuronais pré-treinadas
    \end{itemize}
\end{frame}

\section{Plano de implementação}

\begin{frame}{Setup da base de dados}
    \begin{itemize}
        \item Conteúdo em desenvolvimento...
        \item Instalação e configuração do Weaviate
        \item Requisitos de sistema
        \item Considerações de desempenho
    \end{itemize}
\end{frame}

\begin{frame}{Desenho do esquema da base de dados}
    \begin{itemize}
        \item Conteúdo em desenvolvimento...
        \item Estrutura das classes no Weaviate
        \item Definição de propriedades para dados de áudio
        \item Relações entre entidades
    \end{itemize}
\end{frame}

\begin{frame}{Processamento do áudio}
    \begin{itemize}
        \item Conteúdo em desenvolvimento...
        \item Pré-processamento de ficheiros
        \item Normalização
        \item Segmentação
    \end{itemize}
\end{frame}

\begin{frame}{Recolher os metadados do áudio e converter em vetores}
    \begin{itemize}
        \item Conteúdo em desenvolvimento...
        \item Extração de metadados
        \item Técnicas de vetorização
        \item Abordagens de embedding
    \end{itemize}
\end{frame}

\begin{frame}{Alimentar a base de dados}
    \begin{itemize}
        \item Conteúdo em desenvolvimento...
        \item Métodos de ingestão de dados
        \item APIs e ferramentas
        \item Processamento em lote vs. tempo real
    \end{itemize}
\end{frame}

\begin{frame}{Desenvolver queries ao sistema}
    \begin{itemize}
        \item Conteúdo em desenvolvimento...
        \item Configuração de parâmetros de similaridade
        \item Estrutura das consultas GraphQL
        \item Otimização de resultados
    \end{itemize}
\end{frame}

\begin{frame}{Desenvolver uma GUI?}
    \begin{itemize}
        \item Conteúdo em desenvolvimento...
        \item Opções de interface
        \item Funcionalidades principais
        \item Visualização de resultados
    \end{itemize}
\end{frame}

\section{Desafios e Limitações}

\begin{frame}{Desafios no trabalho com vetores de áudio}
    \begin{itemize}
        \item Variabilidade de qualidade do áudio
        \item Custo computacional do pré-processamento e extração de características
        \item Equilibrar a precisão do vetor com os requisitos de armazenamento/desempenho
    \end{itemize}
\end{frame}



\begin{frame}{Conclusão}
    \begin{itemize}
        \item As bases de dados vetoriais oferecem novas possibilidades para gestão de áudio
        \item Weaviate apresenta o melhor equilíbrio para as nossas necessidades
        \item A representação eficiente de áudio em vetores é um desafio importante
        \item O projeto tem potencial para aplicações em recomendação, pesquisa e organização de conteúdo áudio
    \end{itemize}
\end{frame}

\begin{frame}{Obrigado pela atenção}
    \centering
    \Large Obrigado pela vossa atenção!
    
    \vspace{1cm}
    
    \normalsize
    Questões?
    
    \vspace{1cm}
    
    \small
    Contacto: a100092@alunos.uminho.pt, a100054@alunos.uminho.pt, a97686@alunos.uminho.pt

    

\end{frame}

\end{document}